\documentclass[10pt]{article}

\usepackage[english]{babel}
\usepackage[utf8x]{inputenc}
\usepackage{amsmath}
\usepackage{amssymb}
\usepackage{amsfonts}
\usepackage{graphicx}
\usepackage[ruled,linesnumbered,noend]{algorithm2e}
\usepackage{empheq}
\usepackage{float}
\usepackage{enumitem}
\usepackage{tikz}
\usepackage[colorlinks=true,urlcolor=blue]{hyperref}

\title{Introduction to Machine Learning, Fall 2014 - Exercise session III}
\author{Rodion ``rodde'' Efremov \\ 013593012}

\begin{document}
 \maketitle

\color{blue}
\section*{Problem 1 (3 points)}

\color{blue}
\section*{Problem 2 (3 points)}

\color{blue}
\section*{Problem 3 (3 points)}

\color{blue}
\section*{Problem 4 (15 points)}
In this exercise we implement an extremely simple prototype-based classifier to classify handwritten digits from the MNIST dataset, and compare that to a nearest-neighbor classifier.

\begin{itemize}
\item[(a)] Download the MNIST data from the course web page. In addition to the actual data, the package contains some functions for easily loading the data into Matlab/Octave/R and for displaying digits. See the README files for details. Load the first $N=5,000$ images using the provided function.

\color{black}
First we need to get to the directory containing the \texttt{loadmnist.m} file, run it, and load 5000 images along their class labels.
\begin{verbatim}
cd /path/to/mnist
run('loadmnist.m');
[X y] = loadmnist(5000);
\end{verbatim}
Now the matrix $X$ contains an image at each row (\texttt{X(i,:)} is an $i$th image).

\item[(b)] As to verify the data, we need a randomly selected array of indices:
\begin{verbatim}
indices = randperm(5000, 100);
\end{verbatim}
After that we can draw the digits from the actual data and compare them to the actual labels:
\begin{verbatim}
visual(X(indices,:));
y(indices) # Prints the actual labels. 
           # Appeared to be in accord with the drawn digits.
\end{verbatim}

\end{itemize}

\end{document}